\section{Basics of Vectors and Matrices}
\exercise

要求$\ket{v_1}$和$\ket{v_2}$线性无关,就是要求方程组
\[
\begin{pmatrix}
x & 2 \\
y & x - y \\
3 & 1
\end{pmatrix}
\begin{pmatrix}
c_1 \\ c_2
\end{pmatrix}
= 0
\]
仅存在零解。因此矩阵
\[
\begin{pmatrix}
x & 2 \\
y & x - y \\
3 & 1
\end{pmatrix}
\]
是必须是满秩的。将其做初等行变换
\[
\begin{pmatrix}
3 & 1 \\
0 & x - 6 \\
0 & 3x - 4y
\end{pmatrix}
\]
所以得到$\ket{v_1}$和$\ket{v_2}$线性无关的条件
\[
x \neq 6, \text{且}\ 3x \neq 4y
\]

\exercise

证明三个向量$\{\ket{v_i}\}, i = 1, 2, 3$构成$\mathbb{C}^3$的基,只需要证明它们是线性无关的即可。对矩阵
\[
\begin{pmatrix}
1 & 1 & 1 \\
1 & 0 & -1 \\
1 & 1 & -1
\end{pmatrix}
\]
做初等行变换得到
\[
\begin{pmatrix}
1 & 1 & 1 \\
0 & 1 & 2 \\
0 & 0 & 2
\end{pmatrix}
\]
很明显矩阵是满秩的,说明这组向量是线性无关的。因此向量$\{\ket{v_i}\}, i = 1, 2, 3$构成$\mathbb{C}^3$的基。

\exercise
\[
||\ket{x}|| = \left[1 + 1 + (2^2 + 1)\right]^{1/2} = \sqrt{7}
\]
\[
\Braket{x|y} = (2 - i) -i + (2 - i)(2 + i) = 7 - 2i
\]
\[
\Braket{y|x} = (2 + i) + i + (2 - i)(2 + i) = 7 + 2i
\]

\exercise
\[
\begin{split}
\Braket{x|y} &= \sum\limits_{i=1}^n x_i^* y_i \\
&= \sum\limits_{i=1}^n (x_i y_i^*)^* \\
&= \left(\sum\limits_{i=1}^n y_i^* x_i \right)^* \\
&= \Braket{y|x}^*
\end{split}
\]

\exercise
\[
c_1 = \Braket{e_1|v} = \frac{5}{\sqrt{2}}
\]
\[
c_2 = \Braket{e_2|v} = \frac{1}{\sqrt{2}}
\]

\exercise

(1)
\[
\ket{e_1} = \frac13 (-1, 2, 2)^t
\]
\[
\ket{f_2} = (2, -1, 2)^t
\]
\[
\ket{e_2} = \frac13 (2, -1, 2)^t
\]
\[
\ket{f_3} = (2, 2, -1)^t
\]
\[
\ket{e_3} = \frac13 (2, 2, -1)^t
\]

(2)
\[
c_1 = \Braket{e_1|u} = 3
\]
\[
c_2 = \Braket{e_2|u} = 6
\]
\[
c_3 = \Braket{e_3|u} = -3
\]

\exercise
\[
\ket{e_1} = \frac{\ket{v_1}}{||\ket{v_1}||} = \frac{\sqrt{3}}{3}(1,i,1)^t
\]
\[
\ket{f_2} = (2, 1-i, i-1)^t
\]
\[
\ket{e_2} = 2\sqrt{2}(2, 1-i, i-1)^t
\]

\exercise
\[
[(cA)^\dagger]_{jk} = (cA)_{kj}^* = c^* A_{kj}^* = c^* (A^\dagger)_{jk} \rightarrow (cA^\dagger) = c^* A^\dagger
\]
\[
[(A+B)^\dagger]_{jk} = (A+B)_{kj}^* = A_{kj}^* + B_{kj}^* = (A^\dagger)_{jk} + (B^\dagger)_{jk} \rightarrow (A+B)^\dagger = A^\dagger + B^\dagger
\]
\[
\begin{split}
[(AB)^\dagger]_{jk} &= (AB)_{kj}^* = \Big(\sum\limits_i A_{ki}B_{ij}\Big)^* \\
&= \sum\limits_i A_{ki}^*B_{ij}^* = \sum\limits_i (A^\dagger)_{ik}(B^\dagger)_{ji} \\
&= (B^\dagger A^\dagger)_{jk}
\end{split} \rightarrow (AB)^\dagger = B^\dagger A^\dagger
\]

\exercise
\[
\text{det}
\begin{vmatrix}
-\lambda & \frac{1+i}{\sqrt{2}} \\ \frac{1-i}{\sqrt{2}} & -\lambda
\end{vmatrix}
= 0 \rightarrow \lambda^2 = 1
\]
$A$的本征值为$\pm 1$。对于$\lambda_1=1$
\[
\frac{1}{\sqrt{2}}
\begin{pmatrix}
0 & 1+i \\ 1-i & 0
\end{pmatrix}
\begin{pmatrix}
v_1 \\ v_2
\end{pmatrix}
=
\begin{pmatrix}
v_1 \\ v_2
\end{pmatrix}
\]
再结合$|v_1|^2+|v_2|^2=1$,得到本征矢
\[
\ket{\lambda_1} = \frac{1}{2}(\sqrt{2}, 1-i)^t
\]
对于$\lambda_2=-1$,得到本征矢
\[
\ket{\lambda_2} = \frac{1}{2}(\sqrt{2}, i-1)^t
\]
\[
\Braket{\lambda_1|\lambda_2} = \frac14 (2-2) = 0
\]
\[
\ket{\lambda_1}\bra{\lambda_1} + \ket{\lambda_2}\bra{\lambda_2} = \frac14
\begin{pmatrix}
2 & \sqrt{2}(1+i) \\ \sqrt{2}(1-i) & 2
\end{pmatrix}
+ \frac14
\begin{pmatrix}
2 & -\sqrt{2}(1+i) \\ \sqrt{2}(i-1) & 2
\end{pmatrix}
=
\begin{pmatrix}
1 & 0 \\ 0 & 1
\end{pmatrix}
\]
使得$A$对角化的酉矩阵为
\[
\frac12
\begin{pmatrix}
\sqrt{2} & \sqrt{2} \\
1-i & i - 1
\end{pmatrix}
\]

\exercise

问题(1):首先从$A$的本征值方程出发
\[
A\ket{\lambda_i} = \lambda_i\ket{\lambda_i} \rightarrow \bra{\lambda_i}A^\dagger = \lambda_i^*\bra{\lambda_i} \rightarrow -\bra{\lambda_i}A = \lambda_i^*\bra{\lambda_i}
\]
两边同时与$\ket{\lambda_i}$内积
\[
-\Braket{\lambda_i|A|\lambda_i} = \lambda_i^*\Braket{\lambda_i|\lambda_i} \rightarrow -\lambda_i = \lambda_i^*
\]
显然$\lambda_i$是纯虚数。


问题(2):同样是算符$U$的本征方程
\[
U\ket{\lambda_i} = \lambda_i\ket{\lambda_i} \rightarrow \bra{\lambda_i}U^\dagger = \lambda_i^*\bra{\lambda_i}
\]
将两个式子结合在一起得到
\[
\Braket{\lambda_i|U^\dagger U|\lambda_i} = \lambda_i^*\lambda_i\Braket{\lambda_i|\lambda_i} \rightarrow \Braket{\lambda_i|\lambda_i} = \lambda_i^*\lambda_i\Braket{\lambda_i|\lambda_i} \rightarrow |\lambda|^2 = 1
\]

问题(3):若$A$是Hermitian矩阵,则$A$显然是正规矩阵,本征值为实数。反之,若$A$的本征值为实数,则
\[
(A-\lambda_i)\ket{\lambda_i} = 0 \rightarrow \bra{\lambda_i}(A^\dagger-\lambda_i) = 0
\]
再结合对易关系$[A^\dagger, A]=0$
\[
\Braket{\lambda_i|(A^\dagger-\lambda_i)(A-\lambda_i)|\lambda_i} = \Braket{\lambda_i|(A-\lambda_i)(A^\dagger-\lambda_i)|\lambda_i} = 0
\]
可得到
\[
A^\dagger\ket{\lambda_i} = \lambda_i\ket{\lambda_i}
\]
所以$A^\dagger=A$,$A$是Hermitian矩阵。

\exercise

将矩阵的基$\{1,2,3\}$轮换为$\{2,3,1\}$,得到矩阵
\[
U^\prime =
\begin{pmatrix}
i & 0 & 0 \\ 0 & 0 & i \\ 0 & i & 0
\end{pmatrix}
\]
显然这个矩阵具有块对角的形式,其中$2\times 2$部分是$i\sigma_x$。因此本征值分别为$i, i, -i$,(未归一化的)本征矢分别为
\[
(1,0,0)^t,\quad (0,1,1)^t,\quad (0,1,-1)^t
\]
而$U$的本征矢需要将基重新轮换回去,并且归一化
\[
(0,1,0)^t,\quad \frac{1}{\sqrt{2}}(1,0,1)^t,\quad \frac{1}{\sqrt{2}}(-1,0,1)^t
\]

\exercise

已知$H$是Hermitian矩阵
\[
H\ket{\lambda_j} = \lambda_j\ket{\lambda_j}
\]
容易证明
\begin{align*}
(I+iH)\ket{\lambda_j} &= (1+i\lambda_j)\ket{\lambda_j} \\
(I-iH)\ket{\lambda_j} &= (1-i\lambda_j)\ket{\lambda_j} \\
(I+iH)^{-1}\ket{\lambda_j} &= (1+i\lambda_j)^{-1}\ket{\lambda_j} \\
(I-iH)^{-1}\ket{\lambda_j} &= (1-i\lambda_j)^{-1}\ket{\lambda_j}
\end{align*}
我们考虑$U^{-1}$和$U^\dagger$作用在任意向量上
\[
\begin{split}
U^{-1}\ket{v} &= (I-iH)(I+iH)^{-1}\ket{v} = \sum\limits_j (I-iH)(I+iH)^{-1}\ket{\lambda_j}\Braket{\lambda_j|v} \\
&= \sum\limits_j (1-i\lambda_j)(1+i\lambda_j)^{-1}\ket{\lambda_j}\Braket{\lambda_j|v}
\end{split}
\]
\[
\begin{split}
U^\dagger\ket{v} &= (I+iH)^{-1}(I-iH)\ket{v} = \sum\limits_j (I+iH)^{-1}(I-iH)\ket{\lambda_j}\Braket{\lambda_j|v} \\
&= \sum\limits_j (1+i\lambda_j)^{-1}(1-i\lambda_j)\ket{\lambda_j}\Braket{\lambda_j|v}
\end{split}
\]
显然$U^{-1}\ket{v}=U^\dagger\ket{v}$,所以$U$是酉矩阵。

